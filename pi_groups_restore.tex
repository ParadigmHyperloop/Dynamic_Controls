\documentclass[11pt,a4paper]{scrartcl}\usepackage{amssymb,bm}\usepackage{amsmath}\begin{document}
\begin{equation}\pi(1)=\frac{length}{height }\end{equation}
\begin{equation}\pi(2)=\frac{area}{height ^{2}}\end{equation}
\begin{equation}\pi(3)=\frac{perimeter}{height }\end{equation}
\begin{equation}\pi(4)=\frac{height^{2}\cdot pressure}{force }\end{equation}
\begin{equation}\pi(5)=\frac{volume}{height ^{3}}\end{equation}
\begin{equation}\pi(6)=\frac{viscosity}{force ^{0.5}\cdot density ^{0.5}}\end{equation}
\begin{equation}\pi(7)=\frac{height\cdot density^{0.5}\cdot velocity}{force ^{0.5}}\end{equation}
\begin{equation}\pi(8)=\frac{height^{5}\cdot density\cdot visc_dissip}{force ^{2}}\end{equation}
\begin{equation}\pi(9)=\frac{mass flow}{force ^{0.5}\cdot height \cdot density ^{0.5}}\end{equation}
\begin{equation}\pi(10)=\frac{height\cdot density^{0.5}\cdot soundspeed}{force ^{0.5}}\end{equation}
\begin{equation}\pi(11)=\frac{height^{3}\cdot density\cdot accel}{force }\end{equation}
\begin{equation}\pi(12)=\frac{holeperimeter}{height }\end{equation}
\begin{equation}\pi(13)=\frac{holearea}{height ^{2}}\end{equation}
\end{document}
